\documentclass[11pt]{article}
\renewcommand{\baselinestretch}{2.0}
\usepackage[margin=2in]{geometry} 
\usepackage{amsfonts,amsmath,amssymb}
\begin{document}
\title{Fractions}
A \textbf{fraction} is defined as a ratio of two numbers, where the number at the bottom cannot be equal to zero.
$$\frac{a}{b} \text{where b $\neq$ 0}$$
In a fraction the number at the top is called the 
\textbf{numerator}, and the number at the bottom is called the \textbf{denominator}.
$$\frac{a}{b} = \frac{\textbf{numerator}}{\textbf{denominator}}$$
There are two different kinds of fractions, proper and improper. In a \textbf{proper fraction} the numerator (the number at the top) is less than the denominator (the number at the bottom). In an \textbf{improper fraction}, the numerator is greater than the denominator.  
\centerline{\texbf{Proper Fraction:} numerator < denominator}
\centerline{\texbf{Improper Fraction:} numerator > denominator}

\section{Basic Operations}
\subsection{Addition or subtraction of fractions with the same denominator.}
When you need to add or subtract fractions that contain the same denominator, all you need 
to do is to add or subtract the numerators, depending on the problem, and keep the same 
denominator
$$\frac{a}{b}+\frac{c}{b} = \frac{a+c}{b}$$
Examples $$\text{Addition: } \frac{1}{5}+\frac{2}{5} = \frac{1+3}{5} = \frac{4}{5}\text{ Subraction: } \frac{3}{5}-\frac{2}{5} = \frac{3-2}{5} = \frac{1}{5}$$
\subsection{Addition or subtraction of fractions with different denominators}
To add or subtract fractions with different denominators, we must first find a common number between the two denominators making the problem easier to solve. This number is called 
\textbf{least common denominator (LCD)}.Using the LCD makes the same denominator for all the fractions and the operation will be easier to perform.

\\Suppose we have the following problem: $$\frac{3}{2}+\frac{3}{4}$$

To find the LCD, we must think of a common multiple for both denominators. Our denominators are 2 and 4 their multiples are as follow.

Multiples of 2 : 2 , \textbf{4} , 6  
\\ Multiples of 4 : \textbf{4} , 8  , 12  

As you can see our LCD is 4 since it is the first multiple that both numbers have in common.  

Now that we have found our LCD we can continue on with the operation. Rewrite both fractions to equivalent fractions with 4 as their denominator
$$\frac{3}{2}+\frac{3}{4} = \frac{3\times2}{2\times2} + \frac{3\times1}{4\times1} = \frac{6}{4} + \frac{3}{4} = \frac{9}{4}$$

\subsection{Multiplication of fractions}
In multiplication of fractions, multiply numerators with numerators and denominators with denominators.
$$\frac{3}{5}\times\frac{4}{3} = \frac{3\times4}{5\times3} = \frac{12}{15}$$

\subsection{Division of fractions}
To divide fractions we first need to invert the second fraction and then perform a multiplication of fractions.

$$\frac{3}{4}\div\frac{6}{8} \to \frac{3}{4}\times\frac{8}{6} = \frac{3\times8}{4\times6} = \frac{24}{42}$$  
\end{document}